\documentclass[a4paper,12pt]{article} % This defines the style of your paper

\usepackage[top = 2.5cm, bottom = 2.5cm, left = 2.5cm, right = 2.5cm]{geometry} 
\usepackage[utf8]{inputenc} %utf8 % lettere accentate da tastiera
\usepackage[english]{babel} % lingua del documento
\usepackage[T1]{fontenc} % codifica dei font

\usepackage{multirow} % Multirow is for tables with multiple rows within one 
%cell.
\usepackage{booktabs} % For even nicer tables.

\usepackage{graphicx} 

\usepackage{setspace}
\setlength{\parindent}{0in}

\usepackage{float}

\usepackage{fancyhdr}

\usepackage{caption}
\usepackage{amssymb}
\usepackage{amsmath}
\usepackage{mathtools}
\usepackage{color}

\usepackage[hidelinks]{hyperref}
\usepackage{csquotes}
\usepackage{subfigure}

\usepackage{ifxetex,ifluatex}
\usepackage{etoolbox}
\usepackage[svgnames]{xcolor}

\usepackage{tikz}

\usepackage{framed}

\newcommand*\quotefont{\fontfamily{LinuxLibertineT-LF}} % selects Libertine as 
%the quote font


\newcommand*\quotesize{40} % if quote size changes, need a way to make shifts 
%relative
% Make commands for the quotes
\newcommand*{\openquote}
{\tikz[remember picture,overlay,xshift=-4ex,yshift=-1ex]
	\node (OQ) 
	{\quotefont\fontsize{\quotesize}{\quotesize}\selectfont``};\kern0pt}

\newcommand*{\closequote}[1]
{\tikz[remember picture,overlay,xshift=4ex,yshift=-1ex]
	\node (CQ) {\quotefont\fontsize{\quotesize}{\quotesize}\selectfont''};}

% select a colour for the shading
\colorlet{shadecolor}{WhiteSmoke}

\newcommand*\shadedauthorformat{\emph} % define format for the author argument

% Now a command to allow left, right and centre alignment of the author
\newcommand*\authoralign[1]{%
	\if#1l
	\def\authorfill{}\def\quotefill{\hfill}
	\else
	\if#1r
	\def\authorfill{\hfill}\def\quotefill{}
	\else
	\if#1c
	\gdef\authorfill{\hfill}\def\quotefill{\hfill}
	\else\typeout{Invalid option}
	\fi
	\fi
	\fi}
% wrap everything in its own environment which takes one argument (author) and 
%one optional argument
% specifying the alignment [l, r or c]
%
\newenvironment{shadequote}[2][l]%
{\authoralign{#1}
	\ifblank{#2}
	{\def\shadequoteauthor{}\def\yshift{-2ex}\def\quotefill{\hfill}}
	{\def\shadequoteauthor{\par\authorfill\shadedauthorformat{#2}}\def\yshift{2ex}}
	\begin{snugshade}\begin{quote}\openquote}
		{\shadequoteauthor\quotefill\closequote{\yshift}\end{quote}\end{snugshade}}

\newcommand{\footref}[1]{%
	$^{\ref{#1}}$%
}

\pagestyle{fancy}

\setlength\parindent{24pt}

\fancyhf{}

\lhead{\footnotesize Software Engineering: Assignment 11}

\rhead{\footnotesize Giorgia Adorni}

\cfoot{\footnotesize \thepage} 


\usepackage{xcolor}

\usepackage{listings}
\lstset{
	basicstyle=\scriptsize,
	showstringspaces=false,
	breaklines=true,
	%numbers=left,
	%stepnumber=1,
	commentstyle=\color{red},
	keywordstyle=\color{blue}
}

\lstdefinestyle{DOS}
{
	backgroundcolor=\color{black},
	breaklines=true,
	basicstyle=\scriptsize\color{white}\ttfamily
}

\begin{document}
	\thispagestyle{empty}  
	\noindent{
		\begin{tabular}{p{15cm}} 
			{\large \bf Software Engineering} \\
			Università della Svizzera Italiana \\ Faculty of Informatics \\ 
			\today  \\
			\hline
			\\
		\end{tabular} 
		
		\vspace*{0.3cm} 
		
		\begin{center}
			{\Large \bf Assignment 11: Test Automation}
			\vspace{2mm}
			
			{\bf Giorgia Adorni (giorgia.adorni@usi.ch)}
			
		\end{center}  
	}
	\vspace{0.4cm}
	
	%%%%%%%%%%%%%%%%%%%%%%%%%%%%%%%%%%%%%%%%%%%%%%%%
	%%%%%%%%%%%%%%%%%%%%%%%%%%%%%%%%%%%%%%%%%%%%%%%%
	\section{Introduction}
	The main purpose of this assignment is to analyse the classes \texttt{DirectedAcyclicGraph} and \texttt{EdmondsBlossomShrinking} provided by jgrapht resources, using \textbf{SonarLint}, an IDE extension that helps you detect and fix quality issues as you write code. Then, \textbf{Randoop}, that is a command-line tool used for automatic unit test generation for a given Java program, is used to generate tests with a time budget of 3 seconds.
	
	\section{DirectedAcyclicGraph}
	
	\subsection*{SonarLint}
	Running the analysis on \texttt{DirectedAcyclicGraph} class, \textbf{SonarLint} found the issues shown in Figure \ref{fig:sonar-graph}.
	
	\begin{figure}[H]
		\centering
		\includegraphics[width=\linewidth]{img/DirectedAcyclicGraph/DirectedAcyclicGraph-SonarLint.png}	
		\caption{SonarLint issues founded in \texttt{DirectedAcyclicGraph}}
		\label{fig:sonar-graph}
	\end{figure}

	In total 29 issues are found, in particular, 4 Critical Code Smells, 5 Major Code Smells and 20 Minor Code Smells.
	
	Analysing  the severity scale of \textbf{SonarLint} issues, the most serious problems identified, defined as Critical Code Smells, are those reporting "Make ... transient or serializable". I completely agree with the seriousness associated with these as they violate the contract of the serializable interface causing exceptions at runtime.
	All the minor problems identified are all very useful suggestions to the programmer and they can be followed or not as they do not affect the functioning of the program.
		
	\subsection*{Randoop}
	
	\textbf{Randoop} is run on \texttt{DirectedAcyclicGraph} class with the following command:
	
	\begin{lstlisting}[style=DOS,caption={Randoop Execution}, captionpos=b]
java -classpath lib/randoop-all-4.2.0.jar:out/production/jgrapht-core-0.9.2-sources 
randoop.main.Main gentests --classlist=classlist.txt --time-limit=3
	\end{lstlisting}
	
	Then, \textbf{Randoop} generates an error-revealing test (by default called \\ \texttt{ErrorTest0.java}) that is compiled and executed with the following commands:
	\begin{lstlisting}[style=DOS,caption={Compile Error Revealing Test}, captionpos=b]
javac -classpath lib/randoop-all-4.2.0.jar ErrorTest0.java -sourcepath src/
	\end{lstlisting}
	
	\begin{lstlisting}[style=DOS,caption={Execute Error Revealing Test}, captionpos=b]
java -classpath lib/randoop-all-4.2.0.jar:out/production/jgrapht-core-0.9.2-sources:. 
org.junit.runner.JUnitCore ErrorTest0
	\end{lstlisting}	
	
	After compiling, 9 different tests are generated.
	\begin{figure}[H]
		\centering
		\includegraphics[width=\linewidth]{img/DirectedAcyclicGraph/randoop-execution.png}
		\caption{Compile {Randoop}}
		\label{fig:randexe-graph}
	\end{figure}

After the execution, it is possible to see that all the tests thrown \texttt{NullPointerException}. 

\begin{figure}[H]
	\centering
	\begin{minipage}[b]{.9\textwidth}
		\includegraphics[width=\linewidth]{img/DirectedAcyclicGraph/error-revealing-test1.png}	
	\end{minipage}
	\begin{minipage}[b]{.9\textwidth}
		\includegraphics[width=\linewidth]{img/DirectedAcyclicGraph/error-revealing-test2.png}	
	\end{minipage}
\caption{Randoop execution on \texttt{DirectedAcyclicGraph}}
\label{fig:randtest2-graph}
\end{figure}


\lstinputlisting[language=Java, caption={ErrorTestDirectedAcyclicGraph}, captionpos=b]{errortests/ErrorTestDirectedAcyclicGraph.java}
	
	
All the \texttt{NullPointerExceptions} are caused by a similar sequence of incorrect calls.
Both the methods \texttt{VisitedArrayImpl()} and \texttt{VisitedBitSetImpl()} do not define constructors, except the default one that does not do anything. 
These classes expect firstly a call to the \texttt{getInstance(Region region)} method, that \textbf{Randoop} does not execute. 
Hence, all the attributes are set null and this cause the failure of the calling of any other method.

	\section{EdmondsBlossomShrinking}
	
	\subsection*{SonarLint}
	Running the analysis on \texttt{EdmondsBlossomShrinking} class, \textbf{SonarLint} found the issues shown in Figure \ref{fig:sonar-ed}.
	
	\begin{figure}[H]
		\centering
		\includegraphics[width=\linewidth]{img/EdmondsBlossomShrinking/EdmondsBlossomShrinking-SonarLint.png}	
		\caption{SonarLint issues founded in \texttt{EdmondsBlossomShrinking}}
		\label{fig:sonar-ed}
	\end{figure}
	
	In total 16 issues are found, in particular, 1 Critical Code Smells, 3 Major Code Smells, 10 Minor Code Smells and 2 Info Code Smell.
	
	Analysing  the severity scale of \textbf{SonarLint} issues, contrary to the \texttt{DirectedAcyclicGraph} class, in the \texttt{EdmondsBlossomShrinking} class, the most serious problems identified report "Refactor this method to reduce its Cognitive Complexity", which curiously I would not have identified as a problem of correctness as I did in the case of the violation of the serializable interfaces. 
	Since the code is highly complex, it is difficult to guarantee its correctness and probably a future modification will lead to errors in the code.
	
	The minor errors are the same as those that appear in the analysis of \texttt{DirectedAcyclicGraph}.
		
	\subsection*{Randoop}
	Randoop is run on \texttt{EdmondsBlossomShrinking} class with the following command
	
	\begin{lstlisting}[style=DOS,caption={Randoop Execution}, captionpos=b]
java -classpath lib/randoop-all-4.2.0.jar:out/production/jgrapht-core-0.9.2-sources 
randoop.main.Main gentests --classlist=classlist.txt --time-limit=3
	\end{lstlisting}
	
	Then, \textbf{Randoop} generates an error-revealing test (by default called \\ \texttt{ErrorTest0.java}) that is compiled and executed with the following commands:
	\begin{lstlisting}[style=DOS,caption={Compile Error Revealing Test}, captionpos=b]
javac -classpath lib/randoop-all-4.2.0.jar ErrorTest0.java -sourcepath src/
	\end{lstlisting}
	
	\begin{lstlisting}[style=DOS,caption={Execute Error Revealing Test}, captionpos=b]
java -classpath lib/randoop-all-4.2.0.jar:out/production/jgrapht-core-0.9.2-sources:. 
org.junit.runner.JUnitCore ErrorTest0
	\end{lstlisting}	
	
After compiling, only 1 test is generated.
	\begin{figure}[H]
		\centering
		\includegraphics[width=\linewidth]{img/EdmondsBlossomShrinking/randoop-execution.png}
		\caption{Compile {Randoop}}
		\label{fig:randexe-ed}
	\end{figure}
	
	After the execution, it is possible to see that also, in this case, the test thrown a \texttt{NullPointerException} as for the \texttt{DirectedAcyclicGraph} class.
	In this case, the exception is threw when the instance method of a null object is called.
	
	\begin{figure}[H]
		\centering
		\includegraphics[width=\linewidth]{img/EdmondsBlossomShrinking/error-revealing-test.png}	
		\caption{Randoop execution on \texttt{EdmondsBlossomShrinking}}
		\label{fig:randtest-ed}
	\end{figure}
	
	
	\lstinputlisting[language=Java, caption={ErrorTestEdmondsBlossomShrinking}, captionpos=b]{errortests/ErrorTestEdmondsBlossomShrinking.java}
	
	The cause of the \texttt{NullPointerException} is a sequence of incorrect calls. 
	First of all, the \texttt{EdmondsBlossomShrinking} object is created using a deprecated constructor that does not require a graph and does not initialise the graph field. 
	Then, during the call of the method \texttt{getMatching()}, the method \texttt{vertexSet()} is called on an instance of a null object since the graph is \texttt{null}. 
	
\end{document}
